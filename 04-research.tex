\section*{Publications}

\begin{itemize}[noitemsep, leftmargin=*]
  \item \href{https://ieeexplore.ieee.org/abstract/document/9835216/}{Dynamic Batch Parallel Algorithms for Updating PageRank} (2022)
  \item Shared-Memory Parallel Algorithms for Community Detection in Dynamic Graphs (accepted)
  \item Lock-free Computation of PageRank in Dynamic Graphs (accepted)
\end{itemize}




\section*{Technical Reports}

\begin{itemize}[noitemsep, leftmargin=*]
  \item \href{https://arxiv.org/abs/2403.01261}{GSL-LPA: Fast Label Propagation Algorithm (LPA) for Community Detection with no $\cdots$}
  \item \href{https://arxiv.org/abs/2402.11454}{An Approach for Addressing Internally-Disconnected Communities in Louvain Algorithm}
  \item \href{https://arxiv.org/abs/2401.15870}{DF* PageRank: Improved Incrementally Expanding Approaches for Updating PageRank on $\cdots$}
  \item \href{https://arxiv.org/abs/2401.11415}{A Fast Parallel Approach for Neighborhood-based Link Prediction by Disregarding Large Hubs}
  \item \href{https://arxiv.org/abs/2401.03256}{An Incrementally Expanding Approach for Updating PageRank on Dynamic Graphs}
  \item \href{https://arxiv.org/abs/2312.13936}{GVE-Leiden: Fast Leiden Algorithm for Community Detection in Shared Memory Setting}
  \item \href{https://arxiv.org/abs/2312.08140}{GVE-LPA: Fast Label Propagation Algorithm (LPA) for Community Detection in Shared ...}
  \item \href{https://arxiv.org/abs/2312.04876}{GVE-Louvain: Fast Louvain Algorithm for Community Detection in Shared Memory Setting}
  \item \href{https://arxiv.org/abs/2311.14650}{GVEL: Fast Graph Loading in Edgelist and Compressed Sparse Row (CSR) formats}
  \item \href{https://arxiv.org/abs/2310.18537}{Heuristics for Inequality minimization in PageRank values}
\end{itemize}




\ignore{\section*{Reviews}}

\ignore{
I have worked on fast algorithms for community detection in large graphs. They minimize the use of shared data structures, and use stupidly simple, but amazingly fast hashtables.
}

\ignore{
Well, i don't lose hope from failures and "exceptions" - keep on learning. I was nearing the end of my fifth year as a PhD student. Our conference papers (2) got rejected 2-3 times.

The comments range from:
- Exaggerated fault rates
- Algorithms appear inefficient
- Confusion in fault-tolerance claims, race conditions, convergence, and algorithms
- Given limited insight on relationship of network topology with the results
- Lack of experimental results with realistic batch updates, overhead of FT
- Limited scope of results, lack of clarity on what is new, confusion in terminology

Just submitted both once again.
}

\ignore{
I like working with computer systems and improving their performance - and also in my workflow.

Research scholars in our group spent a lot of time running experiments - most of which is simply spent in loading datsets into memory. I don't just focus on optimizing my target problem, but also the infrastructure needed to support the experiments - faster graph loading, faster graph processing algorithms, and scripts to process output logs.
}
