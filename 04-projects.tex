\subsection{Independent Projects}

During my research work (2021-23), I develop a \href{https://github.com/nodef/nvgraph.sh}{CLI for nvGraph} - a GPU-based graph analytics library by NVIDIA, using CUDA; and a \href{https://github.com/nodef/snap-data.sh}{CLI for SNAP dataset} - a collection of more than 50 large networks. To minimize manual work, I write scripts to \href{https://github.com/javascriptf/script-minify-bibtex}{Shorten author, editor, journal, and booktitle in a BibTeX file}, another to \href{https://github.com/javascriptf/script-publications-list}{Get publications list from DBLP as a table in LaTeX or in BibTeX for references}, one to \href{https://github.com/javascriptf/script-gist-conceal}{Make a gist secret by creating a new secret gist, and deleting the old one}, and another to \href{https://github.com/javascriptf/script-git-sync-file}{Sync a file to a Git repository (such as GitHub Gist) periodically} in Node.js. For a short-term course on Distributed Systems, I design of a simple UDP/datagram based \href{https://github.com/javascriptf/nodejs-time-server}{NTP-like client and server} using Node.js to simulate the effect of transmission delays in a local, intra-continental, and inter-continental network.
food additives to their respective full names.\ignore{In 2015-16, I design a few Windows applications. These include a \href{https://github.com/winp/timer}{Countdown timer} for Windows, a tool to \href{https://github.com/winp/text-hash}{Calculate an 8-bit hash of a given string using XOR+ROR method}, another to \href{https://github.com/winp/process-kill}{Terminate unnecessary/unused processes} in Windows, and one to \href{https://github.com/winp/website-ip}{Determine the IP address of a Website from its URL}.}I also write an \href{https://github.com/javascriptf/nodejs-4004-assembler}{Intel 4004 assembler} in Node.js that converts a plain assembly language file to binary.\ignore{While working at Verizon Hyderabad with Prashant Pathak, we develop a \href{https://github.com/raspberrypif/rfid}{Wireless RFID-based Snacks Ticketing System} using a Raspberry Pi with a Weigand based HID reader.}

\vspace{-3ex}
I also write and maintain a diverse range of Open source packages for Node.js. These include tools to \href{https://github.com/nodef/extra-tunnel}{Tunnel web server from private IP}, perform automated trades with a \href{https://github.com/nodef/extra-fyers}{Javascript interface for FYERS API}, get \href{https://github.com/nodef/extra-quotes}{Insightful quotes on a topic}, and get \href{https://github.com/nodef/extra-license}{SPDX License text}. In 2018-19, I note the release of Google's new WaveNet based Text-To-Speech (TTS) engine. I use this opportunity to write a tool that \href{https://github.com/nodef/wikipedia-tts}{Crawls Wikipedia pages and uploads TTS to YouTube}. These videos are available upon searching ``wikipedia audio article". To support this, I write tools to get \href{https://github.com/nodef/extra-gcpconfig}{GCP}/\href{https://github.com/nodef/extra-awsconfig}{AWS} config, use long text with \href{https://github.com/nodef/extra-googletts}{Google TTS}/\href{https://github.com/nodef/extra-amazontts}{Amazon Polly} through a \href{https://github.com/nodef/extra-tts}{Common tool}, \href{https://github.com/nodef/extra-ffmpeg}{Interface with FFMPEG}, \href{https://github.com/nodef/extra-stillvideo}{Generate still video from an audio and image file}, \href{https://github.com/golangf/porjo--youtubeuploader}{Modify Ian Bishop's YouTube Uploader to handle captions}, and \href{https://github.com/nodef/extra-youtubeuploader}{Interface with the modified YouTube uploader from Node.js}.

\vspace{-3ex}
In 2017-18, I use full-text search in PostgreSQL to build a hand-crafted \href{https://ifct2017.github.io}{Natural Language Query System for Indian Food Composition Tables 2017 (IFCT2017)}. IFCT2017 is published by T. Longvah, R. Ananthan, K. Bhaskarachary, and K. Venkaiah from National Institute of Nutrition (NIN). It contains detailed nutrient composition of 528 key foods in India. I use a PDF to Excel converter to read data in tables of each page, merge them together, fix conversion issues through manual checking, and convert it to CSV. Further, I adjust the units used for each nutrient to grams, and add additional relevant high-level columns. Finally I write a Node.js package to get \href{https://github.com/ifct2017/compositions}{Programmatic access to this Compositions data}. Data from a number of additional tables, including the compositions data, is available in the \href{https://github.com/ifct2017/ifct2017}{Main package}. To support natural language queries, I convert the \href{https://github.com/nodef/pg-english}{English language query to an informal SQL} first, perform Named Entity Recognition (NER) with \href{https://github.com/nodef/map-pg}{Promised Map interface for Table in PostgreSQL}, and then convert \href{https://github.com/nodef/pg-slang}{Informal SQL SELECT to formal SQL}. Testing required constant resetting and provisioning of new databases. For this, I write a script to \href{https://github.com/nodef/heroku-addonpool}{Manage Addon Pool of an App in Heroku}. I also prepare CLI tools to convert \href{https://github.com/nodef/food-e}{E numbers} and \href{https://github.com/nodef/food-ins}{INS numbers} of food additives to their respective full names.

\vspace{-3ex}
I also maintain collections of functions for in-built types in Node.js, such as, \href{https://github.com/nodef/extra-array}{Arrays}, \href{https://github.com/nodef/extra-set}{Sets}, \href{https://github.com/nodef/extra-map}{Maps}, \href{https://github.com/nodef/extra-object}{Objects}, \href{https://github.com/nodef/extra-iterable}{Iterables}, \href{https://github.com/nodef/extra-entries}{Entries}, \href{https://github.com/nodef/extra-lists}{Lists}, \href{https://github.com/nodef/extra-ientries}{IEntries}, \href{https://github.com/nodef/extra-ilists}{ILists}, \href{https://github.com/nodef/extra-sorted-array}{Sorted Arrays}, \href{https://github.com/nodef/extra-array-view}{Array Views}, \href{https://github.com/nodef/extra-boolean}{Booleans}, \href{https://github.com/nodef/extra-bit}{Bits}, \href{https://github.com/nodef/extra-integer}{Integers}, \href{https://github.com/nodef/extra-number}{Numbers}, \href{https://github.com/nodef/extra-string}{Strings}, \href{https://github.com/nodef/extra-bigint}{BigInts}, \href{https://github.com/nodef/extra-function}{Functions}, and \href{https://github.com/nodef/extra-async-function}{Async Functions}. In addition, I extend in-built modules, such as, \href{https://github.com/nodef/extra-fs}{File System (FS)}, \href{https://github.com/nodef/extra-child-process}{Child Process}, \href{https://github.com/nodef/extra-path}{Path}, and \href{https://github.com/nodef/extra-math}{Math}. To support the automated building, testing, and documentation generation, I write and use a special \href{https://github.com/nodef/extra-build}{Build tool} - this requires utilities for processing \href{https://github.com/nodef/extra-javascript-text}{JavaScript}, \href{https://github.com/nodef/extra-jsdoc-text}{JSDoc}, and \href{https://github.com/nodef/extra-markdown-text}{Markdown} text. Building documentation for some packages requires the use of a \href{https://github.com/nodef/extra-typedoc-theme}{Custom TypeDoc theme}. To generate ASCII videos of examples automatically, an \href{https://github.com/nodef/extra-asciinema}{Asciinema-based tool} can be used. To delete old/unused packages automatically, I write a tool to \href{https://github.com/javascriptf/devtools-delete-npm-org-packages}{Delete NPM packages in an organization} using Puppeteer. I originally conceived of maintaining Open source code as a potential business venture - which allows me to make passive money through donations. However, the market was already bustling with utility packages. Despite meticulous documentation and detailed list of functions, differentiation proved elusive.




\subsection{Course Projects at IIIT Hyderabad}

At IIIT Hyderabad, I take courses focusing on the systems side. These include Concurrent Data Structures (CDS), Advanced Computer Networks (ACN), Independent Study (IS), Advanced Computer Architecture (ACA), Software Foundations (SF), Optimizations Methods (OM), Compilers, Principles of Programming Languages (PoPL), Discrete Maths and Algorithms (DMA), Distributed Systems (DS), Software Engineering (SF), and Internals of Application Servers (IAS).

% \vspace{-3ex}
In the course \textbf{\href{https://github.com/iiithf/concurrent-data-structures}{Concurrent Data Structures (CDS)}} taught by late Prof. R. Govindarajulu; I implement \href{https://github.com/javaf/k-compare-single-swap}{k-compare single-swap (KCSS)} - an extension of CAS that enables atomically checking multiple addresses before making an update - as the \textit{course project}, in Java. It is based on the paper \href{https://dl.acm.org/doi/10.1145/777412.777468}{``Nonblocking k-compare-single-swap"} by V. Luchangco, M. Moir, and N. Shavit. As a part of course requirement, I \textit{present} on \href{https://gist.github.com/wolfram77/28da72ab511eacafbd55f3576fb03019}{``DDR, GDDR, HBM SDRAM Memory"}, \href{https://gist.github.com/wolfram77/3507129650f2e56e00da013a7de93ddb}{``Concurrency in Distributed Systems, Leslie Lamport papers"}, and \href{https://gist.github.com/wolfram77/77758eb9f7d393598fc142d9559e5a5e}{``Nonblocking k-compare-single-swap"}. We also do a lot of \textit{exercises} - these were optional, but I enjoy them. These include \textit{mutual exclusion} problems and solutions, such as, \href{https://github.com/javaf/sleeping-barber-problem}{Sleeping Barber Problem}, \href{https://github.com/javaf/dining-philosophers-problem}{Dining Philosopher Problem}, \href{https://github.com/javaf/dekker-algorithm}{Dekker's Algorithm}, \href{https://github.com/javaf/peterson-algorithm}{Peterson's Algorithm}, \href{https://github.com/javaf/bakery-algorithm}{Bakery Algorithm}, \href{https://github.com/javaf/simple-semaphore}{Semaphore}, and a \href{https://github.com/javaf/monitor-example}{Monitor}; \textit{locks} for synchronization, such as, \href{https://github.com/javaf/simple-reentrant-lock}{Simple Reentrant Lock}, \href{https://github.com/javaf/simple-read-write-lock}{Simple Read Write Lock}, \href{https://github.com/javaf/fifo-read-write-lock}{FIFO Read Write Lock}, \href{https://github.com/javaf/tas-lock}{TAS Lock}, \href{https://github.com/javaf/ttas-lock}{TTAS Lock}, \href{https://github.com/javaf/backoff-lock}{Backoff Lock}, \href{https://github.com/javaf/array-lock}{Array Lock}, \href{https://github.com/javaf/clh-lock}{CLH Lock}, \href{https://github.com/javaf/mcs-lock}{MCS Lock}, and \href{https://github.com/javaf/bathroom-lock}{Bathroom Lock}; \textit{concurrent data structures}, such as, \href{https://github.com/javaf/locked-queue}{Locked Queue}, \href{https://github.com/javaf/array-queue}{Array Queue}, \href{https://github.com/javaf/array-stack}{Array Stack}, \href{https://github.com/javaf/backoff-stack}{Backoff Stack}, \href{https://github.com/javaf/elimination-backoff-stack}{Elimination Backoff Stack}, \href{https://github.com/javaf/coarse-set}{Coarse Set}, \href{https://github.com/javaf/fine-set}{Fine Set}, and \href{https://github.com/javaf/optimistic-set}{Optimistic Set}; data structures for \textit{software combining}, such as \href{https://github.com/javaf/combining-tree}{Combining Tree}; \textit{balanced counting networks}, such as, \href{https://github.com/javaf/bitonic-network}{Bitonic Network} and \href{https://github.com/javaf/periodic-network}{Periodic Network}; and a concurrent in-memory \href{https://github.com/javaf/savings-account}{Savings Account}.\ignore{During the course, we study the following research papers: \href{https://gist.github.com/wolfram77/0dc7ef397381b0d0bb33bd38331cb572}{``Nonblocking k-compare-single-swap"}, \href{https://gist.github.com/wolfram77/333f712e250e3ef6fca913771f1c7a9e}{``RISC-V offers simple, modular ISA"}, \href{https://gist.github.com/wolfram77/7e3201aa76545759d284b3ab2d910944}{``Real-world Concurrency"}, \href{https://gist.github.com/wolfram77/a0ed73c64f1954ff831a060be4c23092}{``The Concurrency Challenge"}, \href{https://gist.github.com/wolfram77/88b9d87dfcce95d7fd591f8c77be1c35}{``Data Structures in the Multicore Age"}, \href{https://gist.github.com/wolfram77/c03196475788a7c3d000481dab6010da}{``Software and the Concurrency Revolution"}, \href{https://gist.github.com/wolfram77/cfb8376d29f7d2de04143fc5ce411bc6}{``Turing Lecture - The Computer Science of Concurrency - The Early Years"}, and \href{https://gist.github.com/wolfram77/9e38862624bfb9875dcbaec25471e7e6}{``Solution of a Problem in Concurrent Programming Control"}.}

% \vspace{-3ex}
While studying the course \textbf{\href{https://github.com/iiithf/advanced-computer-networks}{Advanced Computer Networks (ACN)}}, taken by Prof. Shatrunjay Rawat; Ram, Ravi and I design a \href{https://github.com/iiithf/lan-design}{Local Area Network for Personal and Common Internet connections along with Security Cameras at Golf View Apartments}, and a \href{https://github.com/iiithf/wan-design}{Wide Area Network for File sharing, Energy \& Gas metering, Fire alarm, Burglar security between 1 Village \& 2 Apartments} - as the course project. I also present on \href{https://gist.github.com/wolfram77/ace297c1087a99fa2f3549914922bf19}{``Distance Vector Multicast Routing Protocol (DVMRP)"}, \href{https://gist.github.com/wolfram77/b1ea41b91a2971aab383a4f6cf4e6378}{``Submarine cables, Indian Perspective"}, \href{https://gist.github.com/wolfram77/53c17aa4aeb99f2c619202d044d783e7}{``Internet Hierarchy, APNIC"}, and \href{https://gist.github.com/wolfram77/a2f344125aff9657f97e61b7f2219462}{``Request For Comments (RFC)"}, as a part of the requirements. We are also given assignments on reading RFCs, reading on CIDR, Farthest (Hop) traceroute, Switch comparision from various companies, and Firewall comparision from Cisco, Juniper, Checkpoint, Fortinet, Sophos, Barracuda, Sonicwall.

% \vspace{-3ex}
While doing \textbf{Independent Study (IS)} of \textit{Infectious Disease Modeling} under the guidance of Prof. Kishore Kothapalli; I prepare presentations on \href{https://gist.github.com/wolfram77/45a01f935b6a72800af16ddbe0af34e3}{``SIR Model, Disease Modelling, Epidemiology"}, \href{https://gist.github.com/wolfram77/a7a90549c3c36667860d2832c093a72b}{``Medical organizations, USA and India"}, and write an animated demo of \href{https://github.com/orgs/processingf/sir_model}{SIR Model simulation} with Processing - a Java based graphical language and IDE. I also study C++, including \href{https://gist.github.com/wolfram77/e1e304d93e2543ebb616588d00cba795}{Lvalues and Rvalues}, and practice a number of CUDA programming examples, such as, querying \href{https://github.com/cudaf/device-properties}{Device Properties}, \href{https://github.com/cudaf/choose-device}{Choosing a Device}, computing \href{https://github.com/cudaf/vector-sum}{Vector Sum}, \href{https://github.com/cudaf/dot-product}{Dot Product}, \href{https://github.com/cudaf/matrix-multiplication}{Matrix Multiplication}, \href{https://github.com/cudaf/histogram}{Histogram}, and measuring \href{https://github.com/cudaf/malloc-test}{Memory Allocation Performance}.

% \vspace{-3ex}
During the course \textbf{\href{https://github.com/iiithf/advanced-computer-architecture}{Advanced Computer Architecture (ACA)}}, taken by Prof. R. Govindarajulu; I write a \href{https://github.com/vhdlf/cpu_basic}{turing-complete 32-bit CPU} in VHDL with data movement, branch, arithmetic, and logical instructions. It follows the instruction format of Intel x86 processors, where each instruction takes 2 register operands and an optional immediate value. Like x86, this has 16 32-bit registers, a flag register, and an instruction pointer. The memory address is made to be 16-bit for simulation purposes. A program running on the CPU (a simulation/testbench) was able to compute factorial of integers.

% \vspace{-3ex}
In the course \textbf{\href{https://github.com/iiithf/software-foundations}{Software Foundations (SF)}}, by Prof. Venkatesh Chopella; I model \href{https://github.com/htmlf/balanced-sliding-window}{Balanced sliding window} - a protocol used where reliable in-order delivery of packets is required (like TCP) - as a transition system, in Elm. Here, two communicating processes $P$ and $Q$ send packets to each other, through a channel. $P$ is considered to be the main sending process, and $Q$ is sending acknowledgements. Each process has a send and a recieve buffer, and the channel is thought of as having $2$ buffers ($P \rightarrow Q$, $Q \rightarrow P$). We also submit assignments of \href{https://github.com/iiithf/software-foundations/blob/main/Assignments/Coq}{basic and induction based proofs} on Coq - an interactive theorem prover, and model \href{https://github.com/htmlf/merge-sort}{Merge sort as a multi-stage transition system} in Elm.

% \vspace{-3ex}
I also take the course \textbf{\href{https://github.com/iiithf/optimization-methods}{Optimization methods (OM)}} by Prof. C.V. Jawahar. As a part of the assignments, I write Python scripts to get introduced to the basic concepts of \href{https://github.com/python3f/gradient-descent}{Gradient Descent}; perform \href{https://github.com/python3f/spectral-clustering}{Spectral Clustering} - a graph-based data grouping algorithm; try \href{https://github.com/python3f/manifold-learning}{Manifold Learning} - an approach to non-linear dimensionality reduction; learn \textit{data visualization techniques}, such as, \href{https://github.com/python3f/isomap}{Isomap}, \href{https://github.com/python3f/locally-linear-embedding}{Locally linear embedding (LLE)}, and \href{https://github.com/python3f/multidimensional-scaling}{Multi-Dimensional Scaling (MDS)}; and \href{https://github.com/python3f/covid19-estimate}{Estimating COVID-19 new cases and unlockdown date}.

% \vspace{-3ex}
While taking the course \textbf{\href{https://github.com/iiithf/compilers}{Compilers}}, taken by Prof. Suresh Purini; I implement an LLVM based parser in C++ for a \href{https://github.com/cppf/basic-parser}{Simpler version of the BASIC programming language}. It supports comments (with $'$), numbers, strings, variables (with type suffix), expressions, single line IF-ELSE, multiline IF-ELSE-ElSE IF-END IF, FOR-NEXT, WHILE-WEND, DO WHILE-LOOP, and DO-LOOP UNTIL loops, SUB-END SUB subroutines and FUNCTION-END FUNCTION functions, and a number of in-built commands, such as, CLS, INPUT, PRINT, DIM, REDIM, EXIT FOR/WHILE/DO, OPEN, CLOSE, EOF, and LINE INPUT.

% \vspace{-3ex}
In the course \textbf{\href{https://github.com/iiithf/principles-of-programming-languages}{Principles of Programming Languages (PoPL)}} by Prof. Venkatesh Chopella; I implement an \href{https://github.com/haskellc/interpreter}{Interpreter} that parses an S-expression to AST, and then evaluates the AST. We implement it in steps - supporting \href{https://github.com/interpreterz/arithmetic2}{Arithmetic}, \href{https://github.com/interpreterz/lexical}{Lexical}, \href{https://github.com/interpreterz/functional}{Functional}, \href{https://github.com/interpreterz/recursive}{Recursive}, and \href{https://github.com/interpreterz/stores}{Stores} (variables) in both Haskell and Racket - a Scheme-like programming language. We solve assignments for performing a number of operations on \href{https://github.com/racketf/lists-assignment}{Lists}, \href{https://github.com/racketf/lists-quiz}{S-Lists/Expressions}, and \href{https://github.com/racketf/trees-assignment}{Trees} in Racket.

% \vspace{-3ex}
During taking the course \textbf{\href{https://github.com/iiithf/distributed-systems}{Distributed Systems (DS)}} by Prof. Kishore Kothapalli; I write Python scripts for \href{https://github.com/python3f/grpc-unit-conversion}{Unit Conversion with gRPC server and client} and for \href{https://github.com/python3f/grpc-who}{Displaying who is logged in on remote server using gRPC}. In the course \textbf{\href{https://github.com/iiithf/software-engineering}{Software Engineering (SE)}} taught by Prof. Prof. Raghu Reddy; I improve upon an existing \href{https://github.com/javaf/bowling-alley}{Bowling Alley Management System} (in Java). We also do in-class activities on the Design of an Automobile Maintenance Diagnostic System, and a Drawing Editor. While taking the course \textbf{\href{https://github.com/iiithf/internals-of-application-servers}{Internals of Application Servers (IAS)}} managed by Prof. Ramesh Loganathan, I develop a \href{https://github.com/python3f/http-socket-server}{Basic HTTP server}; and a \href{https://github.com/python3f/ftp-socket-server}{Basic FTP server}implementation using sockets in Python. We later use these with Tensorflow and ZeroMQ to design an AI model running platform as a service.

% \vspace{-3ex}
During the Summer of 2019, I work with Prof. Vishal Garg as a part of his ongoing Smart metering project. Here, I learn about using the ESP32 microcontroller through a number of \href{https://github.com/iiithf/esp32-examples}{Examples}. These include using \href{https://github.com/esp32f/storage_fatfs}{FAT FS for file read/write}, \href{https://github.com/esp32f/storage_spiffs}{SPIFFS for file operations}, \href{https://github.com/esp32f/wifi_scan}{Scan and log WiFi networks}, \href{https://github.com/esp32f/wifi_ap}{WiFi access point setup}, \href{https://github.com/esp32f/wifi_sta}{WiFi station setup}, \href{https://github.com/esp32f/wifi_apsta}{WiFi in AP and station mode with AP scanning}, \href{https://github.com/esp32f/sntp_sync}{Internet time synchronization with NTP}, \href{https://github.com/esp32f/timer_alarm}{Periodic alarm using timer group}, \href{https://github.com/esp32f/http_server}{Static website server in access point mode}, \href{https://github.com/esp32f/sensor_sht21}{Read temperature and humidity from SHT21 sensor}, \href{https://github.com/esp32f/mqtt_client}{MQTT client for sending repeated messages}, and \href{https://github.com/esp32f/mqtt_sht21}{SHT21 sensor with MQTT over WiFi}.

% \vspace{-3ex}
The courses \textbf{\href{https://github.com/iiithf/discrete-mathematics-and-algorithms}{Discrete Mathematics and Algorithms (DMA)}} by Prof. Bapi Raju and \textbf{\href{https://github.com/iiithf/queuing-theory}{Queuing Theory (QT)}} by Prof. Prof. Sujit Gujar (part of ACN) are purely theoretical. I also enjoy the online \textbf{\href{https://github.com/iiithf/virology}{Virology}} course by Prof. Vincent Racaniello.




\subsection{Course Projects at NIT Rourkela}

At NIT Rourkela, I take courses on Electronics and Communication Engineering. These include Embedded Computing Systems (ECS), Soft Computing (SC), Digital Image Processing (DIP) and Laboratory, Communicative English (CE), Computer Communication Network (CCN), Mobile Communication (MC) and Laboratory, Operating System (OS), Embedded Systems (ES), Digital Communication (DC) Laboratory, Communication System Design (CSD) Laboratory, Analog Communication Systems (ACS), Control System Engineering (CSE), Digital Signal Processing (DSP) and Laboratory, Microprocessor and Laboratory, Electronics Design (ED) Laboratory, Mathematics 4, Digital Electronics (DE) and Laboratory, Analog Electronics (AE) Laboratory, Electrical Engineering (EE) and Laboratory, Semiconductor Devices (SD), Language Laboratory, Mathematics 3, Networks, Algorithm Design (AD), Numerical Methods (NM) Laboratory, and Basic Electronics (BE) Laboratory.

In the \textbf{\href{https://github.com/nitrece/digital-image-processing-laboratory}{Digital Image Processing (DIP) Laboratory}} by Prof. L.P. Roy; we perform experiments on \href{https://github.com/matlabf/image-bit-planes}{Image Bit Planes}, \href{https://github.com/matlabf/image-transform}{Image Transform}, \href{https://github.com/matlabf/image-histogram}{Image Histogram}, \href{https://github.com/matlabf/image-histogram-equalization}{Image Histogram Equalization}, \href{https://github.com/matlabf/image-homomorphic-filter}{Image Homomorphic Filter}, \href{https://github.com/matlabf/image-smoothing}{Image Smoothing}, \href{https://github.com/matlabf/image-denoise}{Image Denoise}, \href{https://github.com/matlabf/image-compress-rle}{Image Compress RLE}, and \href{https://github.com/matlabf/video-histogram-equalization}{Video Histogram Equalization} using MATLAB. In the \textbf{\href{https://github.com/nitrece/mobile-communication-laboratory}{Mobile Communication (MC) Laboratory}} by Prof. Poonam Singh; we peform experiments on a number of digital modulation schemes - Quadrature Phase Shift Keying (QPSK), Quadrature Amplitude Modulation (QAM), On–Off Keying (OOK), and Pulse Position Modulation (PPM).

During the \textbf{\href{https://github.com/nitrece/digital-communication-laboratory}{Digital Communication (DC) Laboratory}} by Prof. U.K. Sahoo; we perform experiments on \href{https://github.com/matlabf/signal-pcm}{Pulse Code Modulation (PCM) and Demodulation}, \href{https://github.com/matlabf/signal-dpcm}{Differential Pulse Code Modulation (DPCM) and Demodulation}, \href{https://github.com/matlabf/signal-dm}{Delta Modulation (DM) and Demodulation}, and \href{https://github.com/matlabf/signal-adm}{Adaptive Delta Modulation (ADM) and Demodulation} using MATLAB. While taking \textbf{\href{https://github.com/nitrece/communication-system-design-laboratory}{Communication System Design (CSD) Laboratory}} by Prof. Poonam Singh; we design circuits for Amplitude Modulation (AM), Frequency Modulation (FM), Amplitude Shift Keying (ASK), Frequency Shift Keying (FSK), Binary Phase-Shift keying (BPSK), and Pulse Code Modulation (PCM) in MultiSim. For the course \textbf{\href{https://github.com/nitrece/digital-signal-processing-laboratory}{Digital Signal Processing (DSP) Laboratory}} by Prof. Sumit Saha; we practised Convolution, Circular Convolution, Quantized Convolution, Convolution with Finite Impulse Response (FIR) Filter, Discrete Fourier Transform (DFT), Inverse Discrete Fourier Transform (IDFT), and Fast Fourier Transform (FFT) in MATLAB.

In the course \textbf{\href{https://github.com/nitrece/microprocessor}{Microprocessor}} by Prof. S.K. Patra; I prepare presentations on \href{https://github.com/nitrece/microprocessor/blob/main/Presentations/01.%20Register%20Comparision%20of%20Intel%208086%20and%20Intel%2080286.pdf}{``Register Comparision of Intel 8086 and Intel 80286"} and \href{https://github.com/nitrece/microprocessor/blob/main/Presentations/02.%20Register%20Organization%20of%20Intel%208086%20vs%20Intel%20Pentium%204%2064-bit.pdf}{``Register Organization of Intel 8086 vs Intel Pentium 4 64-bit"}. For the \textbf{\href{https://github.com/nitrece/microprocessor-laboratory}{Microprocessor Laboratory}} by Prof. S.K. Patra; we perform experiments on 8051, 8085, and 8086 using a Development Board with in-built Assembler, perform Square Wave generation, implement a Traffic Controller, and a \href{https://github.com/electronicsf/digital-clock}{Digital Clock} for 8086 in assembly language. In the \textbf{\href{https://github.com/nitrece/electronics-design-laboratory}{Electronics Design (ED) Laboratory}} by Prof. Subrata Maiti; we design a \href{https://github.com/electronicsf/current-mirror}{Current Mirror}, \href{https://github.com/electronicsf/low-pass-filter}{Low-pass Filter}, \href{https://github.com/electronicsf/high-pass-filter}{High-pass Filter}, \href{https://github.com/electronicsf/band-pass-filter}{Band-pass Filter}, \href{https://github.com/electronicsf/band-reject-filter}{Band-reject Filter}, \href{https://github.com/electronicsf/push-pull-amplifier}{Push-pull Amplifier}, \href{https://github.com/electronicsf/sawtooth-square-generator}{Sawtooth-square Generator}, \href{https://github.com/electronicsf/binary-divider-2bit}{Binary divider 2bit}, and a \href{https://github.com/electronicsf/dc-voltage-stabilizer}{DC Voltage stabilizer} using MultiSim. During this time, I also design a \href{https://github.com/electronicsf/random-number-generator}{Hardware Random Number Generator} in MultiSim, and an \href{https://github.com/electronicsf/led-box}{Animated LED box} using an AVR microcontroller, Veroboard, LEDs, and a soldering iron.

During the \textbf{\href{https://github.com/nitrece/digital-electronics-laboratory}{Digital Electronics (DE) Laboratory}} by Prof. S.K. Das; we perform Verification of Gates, design Half-Full Adder and Half-Full Subtractor, Mux-Demux using 74153 and 74139, Comparators, BCD to Excess 3 and Excess 3 to BCD, Flip-flop, Counters, and Shift Registers. In the \textbf{\href{https://github.com/nitrece/analog-electronics-laboratory}{Analog Electronics (AE) Laboratory}} by Prof. S.M. Hiremath; we design an RC-coupled MOSFET Amplifier (among other experiments), and design an Adjustable Linear Voltage Regulator as project. During this time I write interactive demos of \href{https://github.com/processingf/trigonometry_graphs}{Trignometric functions}, \href{https://github.com/processingf/mosfet_sim}{NMOS and PMOS Id-Vds, Id-Vgs Simulation}, and \href{https://github.com/processingf/fourier_transform}{Real-time Fourier (and inverse) transform of drawable input signal} in Processing. For the \textbf{\href{https://github.com/nitrece/electrical-engineering-laboratory}{Electrical Engineering (EE) Laboratory}} by Prof. Sanjeeb Mohanty; we conduct Speed Control of DC Motor, among other experiments. In the \textbf{\href{https://github.com/nitrece/numerical-methods-laboratory}{Numerical Methods (NM) Laboratory}} by Prof. B.K. Ojha; we write algorithms for \href{https://github.com/cppf/divided-difference-interpolation}{Divided difference Interpolation}, \href{https://github.com/cppf/forward-difference-interpolation}{Forward difference Interpolation}, \href{https://github.com/cppf/backward-difference-interpolation}{Backward difference Interpolation}, \href{https://github.com/cppf/central-difference-interpolation}{Central difference Interpolation}, \href{https://github.com/cppf/lagrange-interpolation}{Lagrange Interpolation}, \href{https://github.com/cppf/bisection-root}{Bisection Root}, \href{https://github.com/cppf/secant-root}{Secant Root}, \href{https://github.com/cppf/newton-root}{Newton Root}, \href{https://github.com/cppf/fixed-point-root}{Fixed point Root}, \href{https://github.com/cppf/regula-falsi-root}{Regula falsi Root}, \href{https://github.com/cppf/rectangular-integration}{Rectangular Integration}, and \href{https://github.com/cppf/trapezoidal-integration}{Trapezoidal Integration} in Turbo C++. While doing the \textbf{\href{https://github.com/nitrece/basic-electronics-laboratory}{Basic Electronics (BE) Laboratory}} by Prof. S.K. Patra; we learn to use Multimeters, Oscilloscopes, and Function generators.

The courses \textbf{\href{https://github.com/nitrece/embedded-computing-systems}{Embedded Computing Systems (ECS)}} by Prof. D.P. Acharya, \textbf{\href{https://github.com/nitrece/soft-computing}{Soft Computing (SC)}} by Prof. Samit Ari, \textbf{\href{https://github.com/nitrece/digital-image-processing}{Digital Image Processing (DIP)}} by Prof. L.P. Roy, \textbf{\href{https://github.com/nitrece/communicative-english}{Communicative English (CE)}} by Prof. Seemita Mohanty, \textbf{\href{https://github.com/nitrece/computer-communication-network}{Computer Communication Network (CCN)}} by Prof. S.K. Das, \textbf{\href{https://github.com/nitrece/mobile-communication}{Mobile Communication (MC)}} by Prof. Poonam Singh, \textbf{\href{https://github.com/nitrece/operating-systems}{Operating System (OS)}} by Prof. Banshidhar Majhi, \textbf{\href{https://github.com/nitrece/embedded-systems}{Embedded Systems (ES)}} by Prof. D.P. Acharya, \textbf{\href{https://github.com/nitrece/analog-communication-systems}{Analog Communication Systems (ACS)}} by Prof. S.M. Hiremath, \textbf{\href{https://github.com/nitrece/control-systems-engineering}{Control System Engineering (CSE)}} by Prof. T.K. Dan, \textbf{\href{https://github.com/nitrece/digital-signal-processing}{Digital Signal Processing (DSP)}} by Prof. Sumit Saha, \textbf{\href{https://github.com/nitrece/mathematics-4}{Mathematics 4}} by Prof. B.K. Ojha, \textbf{\href{https://github.com/nitrece/electrical-engineering}{Electrical Engineering (EE)}} by Prof. Sanjeeb Mohanty, \textbf{\href{https://github.com/nitrece/semiconductor-devices}{Semiconductor Devices (SD)}} by Prof. M.N. Islam, \textbf{\href{https://github.com/nitrece/mathematics-3}{Mathematics 3}} by Prof. B.K. Ojha, \textbf{\href{https://github.com/nitrece/networks}{Networks}} by Prof. Poonam Singh, and \textbf{\href{https://github.com/nitrece/algorithm-design}{Algorithm Design (AD)}} by Prof. Bibhudatta Sahoo are theory focused.

As mentioned in Section \ref{sec:introduction}, I design a Wireless Sensor Node for \href{https://github.com/nitrece/flood-monitoring}{Flood monitoring} as Final year project, intern at NUS Singapore where I work on \href{https://github.com/nitrece/smart-grid-monitoring}{Wireless data acquisition} for NUS Smart Grid and Visualization for an Underwater ROV, and intern at Tata Elxsi Bangalore where I worked on the design of Upsampler and Downsampler filters for a communication system. During the bachelors programme, I stood first thrice in Programming contests (twice at \href{https://github.com/moocf/code-rage}{NIT Rourkela}, and once at IIT Guwahati), and stood second and forth in Open Software Design contests at \href{https://github.com/moocf/programvare-promoteur}{NIT Rourkela} and \href{https://github.com/moocf/open-soft}{IIT Kharagpur} respectively.




\subsection{Hobby Projects at St. Paul's School}

I first got into computer programming with LOGO in the 4th class. However, I take interest in programming when I learn about the \verb|IF| command in GWBASIC - the idea that computers are decision making machines! I took to learn BASIC in great detail - and eventually bump into graphics in GWBASIC itself. From Debashish, I pick up QuickBASIC 4.0, a brilliant programming environment by Microsoft. On the HP-Compaq desktop computer my father bought (with 256MB RAM and Windows XP installed), or the IBM Thinkpad laptop my father brought home after completing lab experiments, I pour in a lot of time on learning QuickBASIC and writing text/graphics mode DOS programs. My text mode programs include \href{https://github.com/qb40/file-sound}{Playing any file as a sound (frequency steps)}, designing a \href{https://github.com/qb40/file-pack}{4-bit Huffman coding based file packer}, performing \href{https://github.com/qb40/gravity-simulation}{Particle gravity simulation with velocity colored traces}, a simple \href{https://github.com/qb40/sound-creator}{Single tone sound creator}, \href{https://github.com/qb40/file-bytes}{Reading the bytes of a file one-by-one}, viewing the \href{https://github.com/qb40/file-graph}{byte values of a file as a line graph}, finding the \href{https://github.com/qb40/file-byte-sum}{Byte sum of a file to verify its integrity}, writing a tool for \href{https://github.com/qb40/file-encrypt}{Encrypting and decrypting a file}, a tool for helping \href{https://github.com/qb40/port-io-helper}{Find and verify IO ports}, another tool to determine the \href{https://github.com/qb40/keyboard-scancode}{Keyboard scancode of a key}, one for \href{https://github.com/qb40/text-draw}{Making a text-mode user interface or art}, a tool for finding basic \href{https://github.com/qb40/system-information}{System information}, a tiny \href{https://github.com/qb40/chocolate-info}{ASCII-art based chocolate info program}, and the classic \href{https://github.com/qb40/stone-paper-scissor}{Stone paper scissor game vs the CPU}.

QuickBASIC's \verb|SCREEN 13| VGA graphics mode is an 8-bit pixel (256 colors using palette) mode with $320\times200$ screen resolution. One can draw a pixel here by simply writing values to the $0xA000$ memory address (text mode is also similar, but speed is not an issue there). In this mode, I experiment with \href{https://github.com/qb40/3d-experiment}{3d formulae}, use \href{https://github.com/qb40/fun-with-mouse}{Left or right mouse button to draw}, design a small and simple \href{https://github.com/qb40/building-animation}{Animation showing rising building like structures}, a basic \href{https://github.com/qb40/ball-animation}{Bouncing ball animation}, demonstrate \href{https://github.com/qb40/designs}{50 different graphics animations}, develop a \href{https://github.com/qb40/picture-creator}{640x400 Picture creator tool}, and another to \href{https://github.com/qb40/image-effect}{Add an animated effect to an image}. I first learn assembly language with \verb|DEBUG32|, a command-line assembler and disassembler, and then move to \verb|MASM| and \verb|FASM| - I use these to write fast routines to draw to screen or perform mouse handling with Interrupt Service Routines (ISRs), with reference from Ralf Brown's Interrupt list. I also write a \href{https://github.com/qb40/boot-register-view}{Bootloader that shows the values of CPU registers} - which I tested using Bochs, a portable open source IA-32 (x86) PC emulator. Do these remind you of the Windows XP days? You can \href{https://github.com/themepackf/windows-xp}{download its theme} if you like.
