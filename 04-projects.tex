\subsection{Course Projects}

At IIIT Hyderabad, I take courses focusing on the systems side. These include Concurrent Data Structures (CDS), Advanced Computer Networks (ACN), Independent Study (IS), Advanced Computer Architecture (ACA), Software Foundations (SF), Optimizations Methods (OM), Compilers, Principles of Programming Languages (PoPL), Discrete Maths and Algorithms (DMA), Distributed Systems (DS), Software Engineering (SF), and Internals of Application Servers (IAS).

In the course \textbf{\href{https://github.com/iiithf/concurrent-data-structures}{Concurrent Data Structures (CDS)}} taught by late Prof. R. Govindarajulu; I implement \href{https://github.com/javaf/k-compare-single-swap}{k-compare single-swap (KCSS)} - an extension of CAS that enables atomically checking multiple addresses before making an update - as the \textit{course project}, in Java. It is based on the paper \href{https://dl.acm.org/doi/10.1145/777412.777468}{``Nonblocking k-compare-single-swap"} by V. Luchangco, M. Moir, and N. Shavit. As a part of course requirement, I \textit{present} on \href{https://gist.github.com/wolfram77/28da72ab511eacafbd55f3576fb03019}{``DDR, GDDR, HBM SDRAM Memory"}, \href{https://gist.github.com/wolfram77/3507129650f2e56e00da013a7de93ddb}{``Concurrency in Distributed Systems, Leslie Lamport papers"}, and \href{https://gist.github.com/wolfram77/77758eb9f7d393598fc142d9559e5a5e}{``Nonblocking k-compare-single-swap"}. We also do a lot of \textit{exercises} - these were optional, but i enjoy them. These include \textit{mutual exclusion} problems and solutions, such as, \href{https://github.com/javaf/sleeping-barber-problem}{Sleeping Barber Problem}, \href{https://github.com/javaf/dining-philosophers-problem}{Dining Philosopher Problem}, \href{https://github.com/javaf/dekker-algorithm}{Dekker's Algorithm}, \href{https://github.com/javaf/peterson-algorithm}{Peterson's Algorithm}, \href{https://github.com/javaf/bakery-algorithm}{Bakery Algorithm}, \href{https://github.com/javaf/simple-semaphore}{Semaphore}, and a \href{https://github.com/javaf/monitor-example}{Monitor}; \textit{locks} for synchronization, such as, \href{https://github.com/javaf/simple-reentrant-lock}{Simple Reentrant Lock}, \href{https://github.com/javaf/simple-read-write-lock}{Simple Read Write Lock}, \href{https://github.com/javaf/fifo-read-write-lock}{FIFO Read Write Lock}, \href{https://github.com/javaf/tas-lock}{TAS Lock}, \href{https://github.com/javaf/ttas-lock}{TTAS Lock}, \href{https://github.com/javaf/backoff-lock}{Backoff Lock}, \href{https://github.com/javaf/array-lock}{Array Lock}, \href{https://github.com/javaf/clh-lock}{CLH Lock}, \href{https://github.com/javaf/mcs-lock}{MCS Lock}, and \href{https://github.com/javaf/bathroom-lock}{Bathroom Lock}; \textit{concurrent data structures}, such as, \href{https://github.com/javaf/locked-queue}{Locked Queue}, \href{https://github.com/javaf/array-queue}{Array Queue}, \href{https://github.com/javaf/array-stack}{Array Stack}, \href{https://github.com/javaf/backoff-stack}{Backoff Stack}, \href{https://github.com/javaf/elimination-backoff-stack}{Elimination Backoff Stack}, \href{https://github.com/javaf/coarse-set}{Coarse Set}, \href{https://github.com/javaf/fine-set}{Fine Set}, and \href{https://github.com/javaf/optimistic-set}{Optimistic Set}; data structures for \textit{software combining}, such as \href{https://github.com/javaf/combining-tree}{Combining Tree}; \textit{balanced counting networks}, such as, \href{https://github.com/javaf/bitonic-network}{Bitonic Network} and \href{https://github.com/javaf/periodic-network}{Periodic Network}; and a concurrent in-memory \href{https://github.com/javaf/savings-account}{Savings Account}.\ignore{During the course, we study the following research papers: \href{https://gist.github.com/wolfram77/0dc7ef397381b0d0bb33bd38331cb572}{``Nonblocking k-compare-single-swap"}, \href{https://gist.github.com/wolfram77/333f712e250e3ef6fca913771f1c7a9e}{``RISC-V offers simple, modular ISA"}, \href{https://gist.github.com/wolfram77/7e3201aa76545759d284b3ab2d910944}{``Real-world Concurrency"}, \href{https://gist.github.com/wolfram77/a0ed73c64f1954ff831a060be4c23092}{``The Concurrency Challenge"}, \href{https://gist.github.com/wolfram77/88b9d87dfcce95d7fd591f8c77be1c35}{``Data Structures in the Multicore Age"}, \href{https://gist.github.com/wolfram77/c03196475788a7c3d000481dab6010da}{``Software and the Concurrency Revolution"}, \href{https://gist.github.com/wolfram77/cfb8376d29f7d2de04143fc5ce411bc6}{``Turing Lecture - The Computer Science of Concurrency - The Early Years"}, and \href{https://gist.github.com/wolfram77/9e38862624bfb9875dcbaec25471e7e6}{``Solution of a Problem in Concurrent Programming Control"}.}

While studying the course \textbf{\href{https://github.com/iiithf/advanced-computer-networks}{Advanced Computer Networks (ACN)}}, taken by Prof. Shatrunjay Rawat; Ram, Ravi and I design a \href{https://github.com/iiithf/lan-design}{Local Area Network for Personal and Common Internet connections along with Security Cameras at Golf View Apartments}, and a \href{https://github.com/iiithf/wan-design}{Wide Area Network for File sharing, Energy \& Gas metering, Fire alarm, Burglar security between 1 Village \& 2 Apartments} - as the course project. I also present on \href{https://gist.github.com/wolfram77/ace297c1087a99fa2f3549914922bf19}{``Distance Vector Multicast Routing Protocol (DVMRP)"}, \href{https://gist.github.com/wolfram77/b1ea41b91a2971aab383a4f6cf4e6378}{``Submarine cables, Indian Perspective"}, \href{https://gist.github.com/wolfram77/53c17aa4aeb99f2c619202d044d783e7}{``Internet Hierarchy, APNIC"}, and \href{https://gist.github.com/wolfram77/a2f344125aff9657f97e61b7f2219462}{``Request For Comments (RFC)"}, as a part of the requirements. We are also given assignments on reading RFCs, reading on CIDR, Farthest (Hop) traceroute, Switch comparision from various companies, and Firewall comparision from Cisco, Juniper, Checkpoint, Fortinet, Sophos, Barracuda, Sonicwall.

While doing \textbf{Independent Study (IS)} under the guidance of Prof. Kishore Kothapalli, TODO.

During the course \textbf{\href{https://github.com/iiithf/advanced-computer-architecture}{Advanced Computer Architecture (ACA)}}, taken by Prof. R. Govindarajulu; I write a \href{https://github.com/vhdlf/cpu_basic}{turing-complete 32-bit CPU} in VHDL with data movement, branch, arithmetic, and logical instructions. It follows the instruction format of Intel x86 processors, where each instruction takes 2 register operands and an optional immediate value. Like x86, this has 16 32-bit registers, a flag register, and an instruction pointer. The memory address is made to be 16-bit for simulation purposes. A program running on the CPU (a simulation/testbench) was able to compute factorial of integers.

In the course \textbf{\href{https://github.com/iiithf/software-foundations}{Software Foundations (SF)}}, by Prof. Venkatesh Chopella; I model \href{https://github.com/htmlf/balanced-sliding-window}{Balanced sliding window} - a protocol used where reliable in-order delivery of packets is required (like TCP) - as a transition system, in Elm. Here, two communicating processes $P$ and $Q$ send packets to each other, through a channel. $P$ is considered to be the main sending process, and $Q$ is sending acknowledgements. Each process has a send and a recieve buffer, and the channel is thought of as having $2$ buffers ($P \rightarrow Q$, $Q \rightarrow P$). We also submit assignments of \href{https://github.com/iiithf/software-foundations/blob/main/Assignments/Coq}{basic and induction based proofs} on Coq - an interactive theorem prover, and model \href{https://github.com/htmlf/merge-sort}{Merge sort as a multi-stage transition system} in Elm.

I also take the course \textbf{\href{https://github.com/iiithf/optimization-methods}{Optimization methods (OM)}} by Prof. C.V. Jawahar. As a part of the assignments, I write Python scripts to get introduced to the basic concepts of \href{https://github.com/python3f/gradient-descent}{Gradient Descent}; perform \href{https://github.com/python3f/spectral-clustering}{Spectral Clustering} - a graph-based data grouping algorithm; try \href{https://github.com/python3f/manifold-learning}{Manifold Learning} - an approach to non-linear dimensionality reduction; learn \textit{data visualization techniques}, such as, \href{https://github.com/python3f/isomap}{Isomap}, \href{https://github.com/python3f/locally-linear-embedding}{Locally linear embedding (LLE)}, and \href{https://github.com/python3f/multidimensional-scaling}{Multi-Dimensional Scaling (MDS)}; and \href{https://github.com/python3f/covid19-estimate}{Estimating COVID-19 new cases and unlockdown date}.

While taking the course \textbf{\href{https://github.com/iiithf/compilers}{Compilers}}, taken by Prof. Suresh Purini; I implemented an LLVM based parser in C++ for a \href{https://github.com/cppf/basic-parser}{Simpler version of the BASIC programming language}. It supports comments (with $'$), numbers, strings, variables (with type suffix), expressions, single line IF-ELSE, multiline IF-ELSE-ElSE IF-END IF, FOR-NEXT, WHILE-WEND, DO WHILE-LOOP, and DO-LOOP UNTIL loops, SUB-END SUB subroutines and FUNCTION-END FUNCTION functions, and a number of in-built commands, such as, CLS, INPUT, PRINT, DIM, REDIM, EXIT FOR/WHILE/DO, OPEN, CLOSE, EOF, and LINE INPUT.

In the course \textbf{\href{https://github.com/iiithf/principles-of-programming-languages}{Principles of Programming Languages (PoPL)}} by Prof. Venkatesh Chopella; Subhajit has written an interpreter for an arithmetic, lexical, functional, recursive, and stores based languages in Haskell.\footnote{\url{https://github.com/interpreterz/stores}}

%% BREAKNECK
During bachelors programme, Subhajit stood first thrice in Programming contests (twice at \href{https://github.com/moocf/code-rage}{NIT Rourkela}, and once at IIT Guwahati), and stood second and forth in Open Software Design contests at \href{https://github.com/moocf/programvare-promoteur}{NIT Rourkela} and \href{https://github.com/moocf/open-soft}{IIT Kharagpur} respectively.


%% COMPUTER ARCHITECTURE
While studying the course Advanced Computer Architecture (ACA) at IIIT Hyderabad, by Prof. R. Govindarajulu; Subhajit has written a  Subhajit has also written an Intel 4004 assembler in Node.js that converts a plain assembly language file to binary.\footnote{\url{https://github.com/javascriptf/nodejs-4004-assembler}}


%% EMBEDDED
While working at Verizon Hyderabad with Prashant Pathak, Subhajit has developed and wireless RFID-based Snacks Ticketing System using a Raspberry Pi with a Weigand based HID reader.\footnote{\url{https://github.com/raspberrypif/rfid}}


%% COMPILERS



%% DESKTOP PROGRAMMING
Subhajit has designed a few Windows applications. These include a countdown timer for Windows;\footnote{\url{https://github.com/winp/timer}} a tool to calculate an 8-bit hash of a given string using XOR+ROR method;\footnote{\url{https://github.com/winp/text-hash}} another to terminate unnecessary/unused processes in Windows;\footnote{\url{https://github.com/winp/process-kill}} and one to determine the IP address of a Website from its URL.\footnote{\url{https://github.com/winp/website-ip}}

Subhajit first got into computer programming with LOGO in the 4th class. However, he took interest in programming when he learnt about the \verb|IF| command in GWBASIC - the idea that computers were decision making machines surprised him. He took to learning BASIC in great detail - eventually bumping into graphs in GWBASIC itself. From his friend Debashish, he picked up QuickBASIC 4.0, a brilliant programming environment by Microsoft. On the HP-Compaq desktop computer his father bought (with 256MB RAM and Windows XP installed), or the IBM Thinkpad laptop his father brought home after completing lab experiments, Subhajit would pour in a lot of time on learning QuickBASIC and writing text/graphics mode DOS programs. His text mode programs include playing any file as a sound (frequency steps);\footnote{\url{https://github.com/qb40/file-sound}} designing a 4-bit Huffman coding based file packer;\footnote{\url{https://github.com/qb40/file-pack}} performing particle gravity simulation with velocity colored traces;\footnote{\url{https://github.com/qb40/gravity-simulation}} a simple single tone sound creator;\footnote{\url{https://github.com/qb40/sound-creator}} read the bytes of a file one-by-one;\footnote{\url{https://github.com/qb40/file-bytes}} viewing the byte values of a file as a line graph;\footnote{\url{https://github.com/qb40/file-graph}} finding the byte sum of a file to verify its integrity;\footnote{\url{https://github.com/qb40/file-byte-sum}} writing a tool for encrypting and decrypting a file;\footnote{\url{https://github.com/qb40/file-encrypt}} a tool for helping find and verify IO ports;\footnote{\url{https://github.com/qb40/port-io-helper}} another tool to determine the keyboard scancode of a key;\footnote{\url{https://github.com/qb40/keyboard-scancode}} one for making a text-mode user interface or art;\footnote{\url{https://github.com/qb40/text-draw}} a tool for finding basic system information tool;\footnote{\url{https://github.com/qb40/system-information}} a tiny ASCII-art based chocolate info program;\footnote{\url{https://github.com/qb40/chocolate-info}} and the classic stone paper scissor game vs the CPU.\footnote{\url{https://github.com/qb40/stone-paper-scissor}}

QuickBASIC's \verb|SCREEN 13| VGA graphics mode is an 8-bit pixel (256 colors using palette) mode with $320\times200$ screen resolution. One can draw a pixel here by simply writing values to the $0xA000$ memory address (text mode is also similar, but speed is not an issue there). In this mode, Subhajit has experimented with 3d formulae;\footnote{\url{https://github.com/qb40/3d-experiment}} use left or right mouse button to draw;\footnote{\url{https://github.com/qb40/fun-with-mouse}} designed a small and simple animation showing rising building like structures;\footnote{\url{https://github.com/qb40/building-animation}} a basic bouncing ball animation;\footnote{\url{https://github.com/qb40/ball-animation}} demonstrated 50 different graphics animations;\footnote{\url{https://github.com/qb40/designs}} developed a 640x400 Picture creator tool;\footnote{\url{https://github.com/qb40/picture-creator}}; and another to Add an animated effect to an image.\footnote{\url{https://github.com/qb40/image-effect}}. Subhajit first learnt assembly language with \verb|DEBUG32|, a command-line assembler and disassembler, and then moved to \verb|MASM| and \verb|FASM| - he used these to write fast routines to draw to screen or perform mouse handling with Interrupt Service Routines (ISRs), with reference from Ralf Brown's Interrupt list. He also write a bootloader that shows the values of cpu registers before booting - which he tested using Bochs, a portable open source IA-32 (x86) PC emulator.\footnote{\url{https://github.com/qb40/boot-register-view}} Do these remind you of the Windows XP days? You can download its theme if you like.\footnote{\url{https://github.com/themepackf/windows-xp}}


%% NETWORKING
Subhajit has worked on the design of a simple UDP/datagram based NTP-like client and server using JavaScript (Node.js) to simulate the effect of transmission delays in a local, intracontinental, and intercontinental network (\href{https://github.com/javascriptf/nodejs-time-server}{here}).


%% RESEARCH
During his research work, Subhajit has written a tool to shorten author, editor, journal, and booktitle in a BibTeX file;\footnote{\url{https://github.com/javascriptf/script-minify-bibtex}} another to get publications list from DBLP as a table in LaTeX or in BibTeX for references;\footnote{\url{https://github.com/javascriptf/script-publications-list}} one to make a gist secret by creating a new secret gist, and deleting the old one;\footnote{\url{https://github.com/javascriptf/script-gist-conceal}} and another to sync a file to a Git repository (such as GitHub Gist) periodically.\footnote{\url{https://github.com/javascriptf/script-git-sync-file}}


%% OPTIMIZATION METHODS


During the course in Distributed Systems, taught by Prof. Kishore Kothapalli, Subhajit has written Python scripts for unit conversion with gRPC server and client;\footnote{\url{https://github.com/python3f/grpc-unit-conversion}} and for displaying who is logged in on remote server using gRPC.\footnote{\url{https://github.com/python3f/grpc-who}} In the course of Internals of Application Servers, managed by Prof. Ramesh Loganathan, Subhajit developed a basic HTTP server implementation using sockets;\footnote{\url{https://github.com/python3f/http-socket-server}} and a basic FTP server implementation using sockets in Python.\footnote{\url{https://github.com/python3f/ftp-socket-server}}






\footnote{\url{https://github.com/python3f/}}

devtools-delete-npm-org-packages Public
Delete NPM packages in an organization.



Have used full-text search in PostgreSQL to build a hand-crafted natural language query system for IFCT2017 (see URL).

Working on fast graph loading algorithm that beats the PIGO, and propular cuhornet, gunrock. Using mmap, vector instructions for parsing text, and OpenMP.

Worked on a fault-tolerant barrier free algorithm for PageRank computation - working on LPA and Louvain approaches.

Good knowledge of C++ and TypeScript (Node.js).

OpenMP

QuickBASIC programs.

Working on a dynamic graph generator - similar to tmux using UNIX sockets.

Dynamic Leiden algorithm.

Working on a fast neighbor-based similarity algorithm - and simrank.

Developed a fast CUDA-based dynamic PageRank algorithm for a levelwise computation approach. Working on a new dynamic PageRank approach that does not require finding connected components.

Developed parallel community detection algorithms using OpenMP - its faster than existing research. Still in progress, static and dynamic.

Developed CUDA-based Louvain and LPA algorithms for community detection - but currently the performance is bottlenecked by the hashtable. Looking to use some sort of collaboarative approach for populating.


Working on a fast distributed dynamic PageRank using MPI - testing on our local 4-node cluster - partitioning using a BFS-like approach, and exchanging only ranks of affected vertices (update).
















ELM lang.

A bit and a boolean utility library.





devtools-relaxed-read Public
A tool to help read with intermittent breaks (YouTube Shorts).

chrome-emboldened-reading Public
Focused emboldening for reading faster online.

chrome-page-stickers Public
Decorate webpages with stickers.
















\subsection{ESP32 Projects}
\href{https://github.com/orgs/esp32f}{GitHub: esp32f}

During the Summer of 2019, i worked with Prof. Vishal Garg for a smart metering project. As a part of the learning process, i learnt about 

\subsubsection{Storage}
\begin{itemize}
    \item \textbf{storage\_fatfs:} FAT FS for file read/write
    \item \textbf{storage\_spiffs:} SPIFFS for file operations
\end{itemize}

\subsubsection{WiFi}
\begin{itemize}
    \item \textbf{wifi\_scan:} Scan and log WiFi networks
    \item \textbf{wifi\_ap:} WiFi access point setup
    \item \textbf{wifi\_sta:} WiFi station setup
    \item \textbf{wifi\_apsta:} WiFi in AP and station mode with AP scanning
\end{itemize}

\subsubsection{Utilities}
\begin{itemize}
    \item \textbf{sntp\_sync:} Internet time synchronization with NTP
    \item \textbf{timer\_alarm:} Periodic alarm using timer group
    \item \textbf{http\_server:} Static website server in access point mode
\end{itemize}

\subsubsection{Sensors and Communication}
\begin{itemize}
    \item \textbf{sensor\_sht21:} Read temperature and humidity from SHT21 sensor
    \item \textbf{mqtt\_client:} MQTT client for sending repeated messages
    \item \textbf{mqtt\_sht21:} SHT21 sensor with MQTT over WiFi
\end{itemize}
