The world is in the midst of an unprecedented growth of interconnected data, and graph processing systems are expected to play a vital role. Conventional graph algorithms designed for static graphs struggle to efficiently handle the continuous changes and updates that occur within these networks. As these networks grow in complexity, the need for algorithms capable of efficiently analyzing dynamic graph data is increasingly crucial.

My PhD thesis focuses on ``Time and Space Efficient Social Network Algorithms for Static and Dynamic Graphs." I have worked on fast loading of graphs (text format) into memory using memory-mapping, outperforming state-of-the-art. I have also designed fast and memory-efficient algorithms for community detection utilizing (auto-generated) SIMD instructions via the Godbolt online compiler, and on GPUs using CUDA. Our algorithms outperform NVIDIA's cuGraph, while running on the CPU. Our work has been accepted by IPDPS workshops (4), Euro-Par workshops (3), the Euro-Par conference (1), the ICPP conference (1), and the Complex Networks conference (1). Other key outputs include fast algorithms for link prediction, the design of a common framework for dynamic graph algorithms, and techniques to address soft faults in dynamic algorithms.

% Subhajit Sahu is pursuing a doctorate degree in Computer Science and Engineering (Spring 2019 - ongoing) with \href{https://faculty.iiit.ac.in/~kkishore/}{Prof. Kishore Kothapalli} as his thesis advisor, at \href{https://cstar.iiit.ac.in}{Center for Security, Theory, and Algorithmic Research (CSTAR)}, IIIT Hyderabad, India - 500 032. His research is in the design of \href{https://scholar.google.com/citations?user=rfOetTIAAAAJ&hl=en}{Efficient Dynamic Graph algorithms for Social Networks that leverage parallelism available on modern architectures}. He has obtained a \href{https://drive.google.com/file/d/1c9goUgwXkyhsWIOc873RDhTk7nEs-hJ5/view?usp=sharing}{CGPA} of $8.4$ out of $10$.\ignore{Prof. Kishore Kothapalli and Prof. Dip Sankar Banerjee have provided letters of recommendation.}

% \vspace{-3ex}
% Before pursuing \href{https://iiithf.github.io}{PhD at IIIT Hyderabad}, Subhajit has worked at Qualcomm Hyderabad in debugging of Android Graphics driver, and at Verizon Hyderabad in upgrading of Cell Site Manager portal and design of \href{https://github.com/raspberrypif/rfid}{RFID-based Snacks automation system}. During his undergraduate studies in \href{https://nitrece.github.io}{Electronics and Communication Engineering at NIT Rourkela}, he has designed a \href{https://github.com/nitrece/flood-monitoring}{Wireless Sensor Node for Flood monitoring} as his Final year project, interned at NUS Singapore where he worked on \href{https://github.com/nitrece/smart-grid-monitoring}{Wireless data acquisition for NUS Smart Grid} and Visualization for an Underwater ROV, and interned at Tata Elxsi Bangalore where he worked on the design of Upsampler and Downsampler filters for a communication system. Subhajit also contributes open source software. He has designed a \href{https://github.com/ifct2017/ifct2017}{Natural Language Query portal for Indian Food Composition 2017}, a simplified \href{https://github.com/nodef/extra-fyers}{Facade API for FYERS Securities}, \href{https://github.com/nodef/extra-array}{Array} and \href{https://github.com/nodef/extra-string}{String} manipulation utilities, and a \href{https://github.com/nodef/nvgraph.sh}{CLI for nvGraph}.
